\problemname{Maskroskaos}
\noindent
Innan Gösta dog av ålderdom kom vetenskapsmän på hur man kan göra människor odödliga. Hur detta gick till lämnas som en
kluring till läsaren. Gösta har en äng med maskrosor. Efter lång väntan är den äntligen fullvuxen, och han kan nu skapa
maskrosläsk med maskrosorna (se \textit{Maskrosor}). Intressant nog är alla maskrosor inte riktigt samma: han har kommit
fram till att det finns $N$ olika arter av maskrosor i hans äng. Han vill nu självklart veta om vissa maskrosarter skapar
godare läsk än andra. Efter en del möda har Gösta lyckats plocka exakt $2$
stycken maskrosor av varje art. Precis när han skulle börja analysera de, så råkar han snubbla och blanda ihop alla
maskrosor. Katastrof!

Eftersom maskrosorna är väldigt lika visuellt kommer han behöva använda sin pollen-analys-maskin (PAM). 
Denna fungerar genom att låta alla maskrosor i den blomma, och kan sedan analysera pollenet. Den kan då avgöra
antalet unika arter maskrosor som finns i den. Den kan tyvärr inte avgöra om det finns en eller två av en viss art,
eftersom mängden pollen kan variera väldigt mycket från maskros till maskros. Han har numrerat alla sina maskrosor
från $1$ till $2N$. 

För att använda maskinen kan du göra två saker:
\begin{itemize}
  \item Lägga in en maskros i maskinen.
  \item Ta ut en maskros ur maskinen (som redan finns i den).
\end{itemize}

Efter att han lägger in eller tar ut en maskros ur maskinen får han direkt reda på antalet unika arter maskrosor
som finns i den.
Gösta vill nu ha din hjälp att använda maskinen för att hitta vilka par av maskrosor som är av samma art.
Eftersom han är odödlig har han inte bråttom, men vill ändå bli färdig inom "rimlig" tid. Därför kan han acceptera
att du använder den som mest en miljon ($10^6$) gånger.

\section*{Implementation}
Din lösning till problemet måste vara skriven i C++. Den ska inkludera filen \textbf{maskroskaos.h}. 
Du ska \textbf{inte} skapa en main-funtkion. Istället ska du implementera funktionen\\

\begin{itemize}
  \item \texttt{void Solve(int N)}\\
  Denna funktion kallas en gång för varje testfall.
  \begin{itemize}
    \item Parametern $N$ ($1 \leq N \leq 43000$) är antalet arter av maskrosor som Gösta plockat.
  \end{itemize}
  Ditt program kan använda följande funktioner för att lösa problemet.
  \begin{itemize}
    \item \texttt{int Query(int x)}\\
    Där $x$ ($1 \leq X \leq 2N$) är indexet av en de $2N$ maskrosorna. Om $x$ inte finns i maskinen läggs den in. Om $x$ redan finns
    i maskinen tas $x$ ut. Dess returvärde är antalet unika arter av maskrosor i den efter att $x$ lagts in eller tagits ut.
    Om du anropar denna funktion mer än $10^6$ gånger eller anger ett $x$ som inte tillfredställer $1 \leq x \leq 2N$
    får du \textbf{Wrong Answer}.
    \item \texttt{int Answer(int a, int b)}\\
    När du anropar denna funktionen hävdar du att maskrosorna med index $a$ och $b$ är av samma art. Om du gör något av följande händer
    får du \textbf{Wrong Answer}.
    \begin{itemize}
      \item Om \texttt{Answer} anropas med samma par av maskrosor mer än en gång.
      \item $1 \leq a,b \leq 2 \cdot N$ inte tillfredställs.
      \item Maskrosorna med index $a$ och $b$ är av olika art.
      \item Returnerar från \texttt{Solve} och du har inte anropat \texttt{Answer} exakt $N$ gånger. 
    \end{itemize}
  \end{itemize}
\end{itemize}

\textbf{Notera:} funktionerna \texttt{Query} och \texttt{Answer} ska kallas på i funktionen \texttt{Solve}. (Se exempelkoden i \texttt{maskroskaos.cpp})

\section*{Poängsättning}
Din lösning kommer att testas på en mängd testfallsgrupper.
För att få poäng för en grupp så måste du klara alla testfall i gruppen.

\noindent
\begin{tabular}{| l | l | p{12cm} |}
  \hline
  \textbf{Grupp} & \textbf{Poäng} & \textbf{Gränser} \\ \hline
  $1$    & $6$         & $N \leq 100$  \\ \hline
  $2$    & $25$        & $N \leq 15000$, för varje par $a$,$b$ där $a$ och $b$ är av samma art gäller det att
  $1 \leq a \leq N$ och $N+1 \leq b \leq 2N$. \\ \hline
  $3$    & $9$         & $N \leq 15000$ \\ \hline
  $4$    & $30$        & $N \leq 38000$ \\ \hline
  $5$    & $5$         & $N \leq 39000$ \\ \hline
  $6$    & $5$         & $N \leq 40000$ \\ \hline
  $7$    & $5$         & $N \leq 41000$ \\ \hline
  $8$    & $5$         & $N \leq 42000$ \\ \hline
  $4$    & $10$        & $N \leq 43000$ \\ \hline
\end{tabular}

\section*{Testning av lösning}
För att underlätta testningen av ditt program tillhandahåller vi ett exempelprogram och en exempeldomare. Dessa kan laddas ned
vid attachments längst ned på sidan. 

För att testa programmet kan du köra följande kommando:
\begin{itemize}
  \item g++ -o grader grader.cpp maskroskaos.cpp
\end{itemize}

Detta kommer skapa filen grader, som du sedan kör för att testa programmet. Notera att denna exempeldomare skiljer sig från
domaren som används för uppgiften. Det är dock garanterat att den riktiga domaren \textbf{inte} är adaptiv. Detta innebär
att svaret bestäms innan ditt program körs, och den kan inte ändra svaret under körningens gång. Ditt program får \textbf{inte}
skriva eller läsa från standard input/output. Om du gör detta kommer du troligtvis få förvirrande fel.

\section*{Indataformat för exempeldomaren}
Exempeldomaren läser indata på följande format:
\begin{itemize}
  \item En rad med heltalet $N$.
  \item $N$ rader som vardera innehåller heltalen $a$ och $b$ ($1 \leq a,b \leq 2N$). Detta innebär att maskrosorna med
  index $a$ och $b$ är av samma art. 
\end{itemize}

\section*{Utdataformat för exempeldomaren}
Om ditt program är rätt skriver den ut "Accepted. Queries: $q$", där $q$ är antalet gånger ditt program anropade \texttt{Query}.

Om ditt program är inkorrekt skriver den ut "WA. $e$", där $e$ är felet ditt program gjorde. Om ditt program har flera
fel kommer den skriva ut det första felet du gör.

\section*{Exempel-interaktion}

Betrakta följande indata till domaren:
\begin{verbatim}
3
1 5
2 4
3 6
\end{verbatim}

En giltig interaktion som löser problem kan se ut som följande:

\begin{table}[htbp]
  \centering
  \begin{tabular}{|c|c|}
      \hline
      \textbf{Call} & \textbf{Return} \\
      \hline
      Solve(3) & (none) \\
      \hline
      Query(1) & 1 \\
      \hline
      Query(2) & 2 \\
      \hline
      Query(5) & 2 \\
      \hline
      Query(1) & 2 \\
      \hline
      Answer(1, 5) & (none) \\
      \hline
      Answer(2, 4) & (none) \\
      \hline
      Answer(3, 6) & (none) \\
      \hline
  \end{tabular}
\end{table}


