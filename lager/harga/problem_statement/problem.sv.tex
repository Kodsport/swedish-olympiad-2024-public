\problemname{Hårgalåten}
\noindent

Hårgalåten är en folkvisa som handlar om hur djävulen får ett gäng ungdomar att dansa
tills de dör. Tom gillar Hårgalåten, och har därför skrivit ett arrangemang som hans
orkester ska spela. För att det ska passa med temat i texten har Tom skrivit 
arrangemanget på ett sätt som gör att man kan fastna för evigt om man börjar på fel ställe.

Orkestern består av $N$ musiker som sitter i en cirkel, där musiker $1$ sitter till vänster om 
musiker $2$, som sitter till vänster om musiker $3$, o.s.v. Notera att den som sitter 
till höger om musiker $N$ är $1$. Arrangemanget består av $M$ olika takter numrerade från $1$ till $M$. 
Från början finns det en lista med heltal $X_1, X_2, \dots, X_N$,
där $X_i$ är vilken takt musiker $i$ ska börja spela. Att spela en takt tar en sekund, och alla
spelar såklart samtidigt. När musiker $i$ har spelat klart takt $X_i$, så tittar hen på vilken 
takt musikern till höger just spelade (alltså $X_{i+1}$). Därefter spelar $i$ takten $f(X_{i+1})$,
där $f(1), f(2), \dots, f(M)$ är en given funktion. Detta upprepas sedan om och om igen.

Med andra ord, om musiker $i$ spelade takt $x$ vid sekund $t$, och musikern till höger spelade takt
$y$ vid sekund $t$, så kommer musiker $i$ spela takt $f(y)$ vid sekund $t+1$. 

Låten tar slut när alla musiker någon gång har spelat alla takter. Din uppgift är att räkna 
ut, för varje musiker, när hen har spelat alla olika takter i låten minst en gång. Det kan hända 
att någon musiker aldrig spelar alla takter, i så fall blir orkestern tvungen att spela för evigt.

\section*{Indata}
Första raden innehåller två heltal $N$ och $M$ ($1 \leq N,M \leq 3 \cdot 10^5$), antalet 
musiker och antalet takter.

Andra raden innehåller $N$ heltal $X_1, X_2, \dots, X_N$ ($1 \leq X_i \leq M$), vilken takt varje musiker ska 
spela vid sekund $1$.

Tredje raden innehåller $M$ heltal $f(1), f(2), \dots, f(M)$ ($1 \leq f(i) \leq M$), funktionen som 
avgör vilka takter musikerna ska spela.

\section*{Utdata}
Om någon musiker aldrig kommer spela alla takter, skriv ut $-1$.
Annars, skriv ut $N$ heltal $t_1, t_2, \dots, t_N$, där $t_i$ är antal sekunder det tar för musiker $i$ att spela alla takter.

\section*{Poängsättning}
Din lösning kommer att testas på en mängd testfallsgrupper.
För att få poäng för en grupp så måste du klara alla testfall i gruppen.

\noindent
\begin{tabular}{| l | l | p{12cm} |}
  \hline
  \textbf{Grupp} & \textbf{Poäng} & \textbf{Gränser} \\ \hline
  $1$    & $10$       & $N = 1$ \\ \hline
  $2$    & $17$       & $f(i) = i$ för alla $1 \leq i \leq M$. \\ \hline
  $3$    & $16$       & $N,M \leq 300$ \\ \hline
  $4$    & $22$       & $N,M \leq 3000$ och $f(i) \neq f(j)$ om $i \neq j$. \\ \hline
  $5$    & $9$        & $N,M \leq 3000$  \\ \hline
  $6$    & $26$       & Inga ytterligare begränsningar. \\ \hline
\end{tabular}


