\problemname{Köpa fika}
\noindent

Ända sedan lilla Kalle vann en gratis biltvätt på trisslott har han börjat se livet på ett annat sätt.
Han har nämligen börjat märka att han har ofantligt mycket tur i vardagen.
En bråkdel av de extraordinära händelser han varit med om är att ha vunnit stadens årliga lathetstävling
efter att ha försovit sig, kommit i tid till skolan varje dag trots att han alltid tagit
sista bussen och fått $1,4$ av $2$ poäng i högskoleprovet genom att kryssa i slumpmässiga svar.

Han tänker nu använda sin ofantliga tur för att hjälpa riksdagen: han tänker nämligen lösa en instans
av det ökända knapsack-problemet. Riksdagen har en fikabudget på $C$ kronor och deras fikaleverantör
har ett utbud av $N$ olika godispåsar. Påse $i$ har ett mått på hur smaskig den är,
$s_i$, och en kostnad på $c_i$ kronor. Varje godispåse kan bara köpas en gång.
Målet är att köpa en delmängd av alla påsar, så att totala kostnaden inte överstiger $C$, samtidigt som
totala nivån av smaskighet är så stor som möjligt. I allmänhet är detta ett väldigt svårt problem att lösa,
men han hoppas att kunna lösa det snabbt genom att nyttja sin tur. 

Hans idé är följande: först, ta alla påsarna och ordna dem i en slumpmässig ordning. Sedan, ignorera
de $K$ första påsarna. Därefter kommer han att betrakta varje påse i ordning och köpa den om han har
råd, annars skippa den och gå vidare. Han fortsätter så ända till han har betraktat alla $N-K$ återstående
godispåsar. 

Om han har tur med sin valda ordning kommer något $K$ att ge det optimala svaret. Kalle har efter mycket slit
ändrat ordningen på alla påsar, och han ber nu dig om hjälp med att utföra hans algoritm för varje $K=0,1,\dots,N-1$.

\section*{Indata}
Första raden innehåller två heltal $N$ och $C$ ($1 \leq N \leq 2 \cdot 10^5$, $1 \leq C \leq 10^9$), antalet godispåsar 
och riksdagens fikabudget i kronor.

Den andra raden innehåller $N$ heltal $s_1, s_2, \dots, s_N$ ($1 \leq s_i \leq 10^9$), smaskigheten av godispåse $i$.

Den tredje raden innehåller $N$ heltal $c_1, c_2, \dots, c_N$ ($1 \leq c_i \leq 10^9$), hur många kronor godispåse $i$
kostar.

Notera att ordningen av godispåsar \textit{inte} är vald likformigt slumpmässigt. 

\section*{Utdata}
Skriv ut $N$ heltal separerade av mellanslag, totala smaskigheten av delmängden godispåsar som Kalles
algoritm hittar för $K=0,1,\dots,N-1$ i ordning. 

\section*{Poängsättning}
Din lösning kommer att testas på en mängd testfallsgrupper.
För att få poäng för en grupp så måste du klara alla testfall i gruppen.

\noindent
\begin{tabular}{| l | l | p{12cm} |}
  \hline
  \textbf{Grupp} & \textbf{Poäng} & \textbf{Gränser} \\ \hline
  $1$    & $9$        & $N \leq 1000$ \\ \hline
  $2$    & $12$       & $C \leq 50$ \\ \hline
  $3$    & $11$       & $c_i \leq c_{i+1}$ för alla $i=1,2,\dots,N-1$. \\ \hline
  $4$    & $17$       & $c_i$ är valt likformigt slumpmässigt \footnote{Se https://en.wikipedia.org/wiki/Discrete\_uniform\_distribution} mellan $1$ och $C$. \\ \hline
  $5$    & $51$       & Inga ytterligare begränsningar. \\ \hline
\end{tabular}


