\problemname{Buying Fika}

\noindent
Ever since little Kalle won a free car wash from a scratch-off ticket, he has started to see life differently.
He has begun to notice that he has an immense amount of luck in everyday situations.
A fraction of the extraordinary events he has recently experienced includes winning the city's annual laziness
competition due to him oversleeping, arriving on time at school every day despite taking the last bus,
and scoring $1.4$ out of $2$ points on the university entrance exam by randomly selecting answers.

He now intends to use his immense luck to help the parliament: he plans to solve an instance of the infamous knapsack problem.
The parliament has a fika budget of $C$ Swedish kronor, and their fika supplier has a selection of $N$ different candy bags.
Each bag $i$ has a measure of how delicious it is, $s_i$, and a cost of $c_i$ kronor. You can only buy each candy bag once.
The goal is to purchase a subset of all the bags, such that the total cost does not exceed $C$, while maximizing the total
level of deliciousness. In general, this is a very difficult problem to solve, but he hopes to solve it quickly by
leveraging his luck.

His idea is as follows: first, take all the bags and arrange them in a random order. Then, ignore the first $K$ bags.
After that, he will consider each bag in order and buy it if he can afford it, otherwise skip it and move on.
He continues this process until he has considered all $N-K$ candy bags.

If he is lucky with his chosen order, some $K$ will give the optimal solution. After a lot of effort, Kalle has changed the order of
all the bags, and he now asks for your help in executing his algorithm for each $K=0,1,\dots,N-1$.

\section*{Input}
The first line of input contains two integers $N$ and $C$ ($1 \leq N \leq 2 \cdot 10^5$, $1 \leq C \leq 10^9$), the
number of candy bags and the parliament's fika budget in kronor.

The second line of input contains $N$ integers $s_1, s_2, \dots, s_N$ ($1 \leq s_i \leq 10^9$), the deliciousness of candy bag $i$.

The third line contains $N$ integers $c_1, c_2, \dots, c_N$ ($1 \leq c_i \leq 10^9$), the price of candy bag $i$ in kronor.

Note that the order of candy bags is \textit{not} chosen uniformly randomly. 


\section*{Output}
Output $N$ integers separated by spaces, the total deliciousness of the subset of bags that Kalle's
algorithm finds for $K=0,1,\dots,N-1$ in order. 

\section*{Points}
Your solution will be tested on several test case groups.
To get the points for a group, it must pass all the test cases in the group.

\noindent
\begin{tabular}{| l | l | p{12cm} |}
  \hline
  \textbf{Group} & \textbf{Point value} & \textbf{Constraints} \\ \hline
  $1$    & $9$        & $N \leq 1000$ \\ \hline
  $2$    & $12$       & $C \leq 50$ \\ \hline
  $3$    & $11$       & $c_i \leq c_{i+1}$ for all $i=1,2,\dots,N-1$. \\ \hline
  $4$    & $17$       & $c_i$ is chosen uniformly randomly \footnote{See https://en.wikipedia.org/wiki/Discrete\_uniform\_distribution} between $1$ and $C$. \\ \hline
  $5$    & $51$       & No additional constraints. \\ \hline
\end{tabular}

%\section*{Samples}
