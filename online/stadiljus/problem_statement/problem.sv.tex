\problemname{Stad i ljus}
För att göra staden mer trivsam har Lunds kommun bestämt sig för att sätta upp ljusstrimmor över gatorna.
Kommunen föredrar att så många platser som möjligt får belysning.
Det finns många platser som ljusstrimmor skulle kunna sättas upp och då kommunfullmäktige är upptagen
har du blivit tillfrågad att lista ut vart ljusstrimmorna ska sättas upp.

\illustration{0.4}{ljusstrimmor}{Ljusstrimmor i Lund, används med tillåtelse}


Gatorna i Lund är olika breda, vilket betyder att ljusstrimmorna kommer behövas specialbeställas.
Det går att specialbeställa ljusstrimmor i längder av hela meter, som då kostar $x$ kr per meter.
För att inte verka slösaktiga kräver kommunen att den genomsnittliga kostnaden för varje ljusstrimma
som beställdes högst får vara $y$ kr per ljusstrimma.

Kommunen har nu gett dig en lista på hur långa ljusstrimmorna behöver vara för att kunna sättas upp
på $N$ olika ställen i staden. Kan du hjälpa dem sätta upp så många som möjligt?

\section*{Indata}
Den första raden i indatan innehåller heltalet $N$ ($1 \leq N \leq 10^5$), antalet platser där ljusstrimmor skulle kunna sättas upp.

Den andra raden innehåller heltalet  $x$ ($1 \leq x \leq 30$), priset på ljusstrimmor i kr/m.

Den tredje raden innehåller heltalet $y$ ($1 \leq y \leq 1000$), det högsta genomsnittliga priset kommunen är villig att betala för ljusstrimmorna.

Därefter följer en rad $N$ heltal $L_1, L_2, \dots, L_n$ ($1<=L_i<=20$), där $L_i$ betyder att ljusstrimman måste vara exakt $L_i$ meter för att belysa den $i$:te platsen.

\section*{Utdata}
Skriv ut ett heltal: det största antalet platser som kan belysas med hjälp av av ljusstrimmor inom kommunens budget.


\section*{Poängsättning}
Din lösning kommer att testas på en mängd testfallsgrupper.
För att få poäng för en grupp så måste du klara alla testfall i gruppen.

\noindent
\begin{tabular}{| l | l | p{12cm} |}
  \hline
  \textbf{Grupp} & \textbf{Poäng} & \textbf{Gränser} \\ \hline
  $1$    & $20$       & $L_i=10$ för alla $i$ \\ \hline
  $2$    & $20$       & $L_i=10$ eller $20$ för alla $i$ \\ \hline
  $3$    & $20$       & $N \leq 500$ \\ \hline
  $4$    & $40$       & Inga ytterligare begränsningar. \\ \hline
\end{tabular}

\section*{Förklaring av exempelfall 1}
Det är möjligt att köpa de första 3 ljusstrimmorna.
Den genomsnittliga konstnaden blir då $\frac{6+9+12}{3}=9$ kr, vilket vi precis klarar inom budgeten av som mest $9$ kr per ljusstrimma i genomsnitt.
