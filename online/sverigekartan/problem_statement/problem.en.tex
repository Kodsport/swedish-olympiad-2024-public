\problemname{Map of Sweden}
Vidar wants to know how large Sweden is. After a lot of effort, he has managed to launch a satellite into the atmosphere that can take several low-resolution
images of Sweden for him. Unfortunately, the satellite's camera is very poor. Each square in the image will either consist of ``\texttt{\#}'',
which means the square contains land, or ``\texttt{.}'' if it contains water.

Vidar is a huge scaredy-cat when it comes to swimming, so in his opinion, Sweden's landmass is only the number of land squares that can be
reached from Stockholm without ever swimming. More precisely, it is the number of squares that can be reached from Stockholm by moving upward,
rightward, downward, or leftward without ever walking over a square that contains water. Note that the square containing Stockholm,
\textbf{\texttt{S}}, also counts as land.

Vidar now wants your help to write a program that calculates Sweden's landmass given a satellite image where he has marked the location of
Stockholm with an \textbf{\texttt{S}}. Then, he will wait for the satellite to take pictures from new angles. By looking at the new images,
he sometimes realizes that a square that used to be water is actually land. When he enters this change into your program, you should
immediately print new landmass.

\section*{Input}
The first line of the input contains an integer $R$ ($1 \le R \le 500$), the number of rows in Vidar's image.

The next line contains an integer $C$ ($1 \le C \le 500$), the number of columns in Vidar's image.

The next line contains an integer $U$ ($0 \le U \le 100,000$), the number of times Vidar changes a square from being water to land.

Then follow $R$ lines, each containing $C$ letters, describing the satellite image Vidar is working with. Each letter is either
``\texttt{\#}'', ``\texttt{.}'' or ``\textbf{\texttt{S}}''. Exactly one square will contain the letter \textbf{\texttt{S}}.

If $U$ is not 0, $U$ lines will follow. Each one will contain two integers $i, j$ ($1 \le i \le R$, $1 \le j \le C$),
which means the square in the $i$-th row and $j$-th column is changed from water to land.

\section*{Output}
First, print an integer: Sweden's total landmass.
Then print $U$ integers: Sweden's total landmass after each time Vidar changes the image.

\section*{Points}
Your solution will be tested on several test case groups.
To get the points for a group, it must pass all the test cases in the group.

\noindent
\begin{tabular}{| l | l | p{12cm} |}
  \hline
  \textbf{Group} & \textbf{Point value} & \textbf{Constraints} \\ \hline
  $1$    & $20$       & $U=0$ and the subtask consist of a single testcase, the one on our poster (\url{https://www.progolymp.se/2024/affisch.pdf}). \\ \hline
  $2$    & $13$       & $R=1, U=0$ \\ \hline
  $3$    & $9$        & $R=1$ \\ \hline
  $4$    & $26$       & $U=0$ \\ \hline
  $5$    & $13$       & $R,C \le 10$ \\ \hline
  $6$    & $19$       & No additional constraints. \\ \hline
\end{tabular}

\section*{Explanation of sample 1}
At first, it's possible to reach 8 land squares from Stockholm. Since Stockholm is also counted as land, Sweden's landmass is 9. After the first update, the grid looks like this:

\noindent
\begin{verbatim}
  .#...#
  .###..
  .#S##.
  ..##..
  ......
\end{verbatim}

\noindent
Since you can now reach a new piece of land, the answer becomes 10. After the next two updates, you can't reach the new land, so the answer does not change.
After the final update, you can reach the newly created land, and the map looks like this:

\noindent
\begin{verbatim}
  .#...#
  .###..
  .#S##.
  ..###.
  ....##
\end{verbatim}