\problemname{Trafikverkets plan}

\noindent
Under vinterhalvåret sker det på tok för många krockar i Lönnköping på grund av hala vägar. 
Eftersom Lönnköpings kommun ville verka ännu effektivare än kattriket Katens erövring (se \textit{Segmentövervakning})
har deras vägnätverk endast ett fåtal vägar. Mellan de $N$ husen finns endast $N-1$ vägar.
Varje väg går direkt mellan två hus. Om du befinner dig vid ett visst hus kan du åka till vilken
annan av vägarna den är kopplad till som helst. Nätverket är dessutom konstruerat så att
man kan ta sig mellan varje par av hus genom att färdas längs med en eller flera vägar.
Trafikverket har nu en briljant plan för att minska antalet krockar: de tänker enkelrikta
varje väg i vägnätverket, så att alla kan ta sig till jobbet på morgonen. På kvällen
kommer de att vända på alla riktningar så att alla kan ta sig hem. 

Det enda problemet är att de inte vet om deras plan kommer att funka. Tänk om någon invånare
inte kan ta sig till jobbet på morgonen? Trafikverket har samlat ihop en lista av var varje
invånare bor och jobbar, och vill nu att du för varje invånare ska avgöra om de kan ta sig
till jobbet givet deras nya plan. Deras plan består alltså av beskrivningen av vägnätverket och
vilken riktning varje väg riktar sig åt.

\section*{Indata}
Den första raden i indatan innehåller heltalet $N$ ($1 \le N \le 3 \cdot 10^5$), antalet hus
som finns i Lönnköping.

De följande $N-1$ raderna innehåller vardera heltalen $a,b$ ($1 \leq a,b \leq N$, $a \neq b$),
vilket betyder att det finns en riktad väg från hus $a$ till hus $b$. Det är garanterat
att om vägarna var oriktade hade man kunnat färdas mellan varje par av hus endast genom att använda
vägarna.

Därefter följer en rad som innehåller heltalet $Q$ ($1 \leq Q \leq 10^5$), antalet invånare i
staden.

De följande $Q$ raderna innehåller vardera heltalen $H_i,W_i$ ($1 \leq H_i,W_i \leq N$), vilket hus
den $i$:te personen bor i respektive vilket hus de jobbar i.

\section*{Utdata}
För varje person, skriv ut ``ja'' om de kan ta sig till jobbet under den nya planen, annars
``nej''. Skriv ut dessa i samma ordning som indatan.

\section*{Poängsättning}
Din lösning kommer att testas på en mängd testfallsgrupper.
För att få poäng för en grupp så måste du klara alla testfall i gruppen.

\noindent
\begin{tabular}{| l | l | p{12cm} |}
  \hline
  \textbf{Grupp} & \textbf{Poäng} & \textbf{Gränser} \\ \hline
  $1$    & $10$       & $N, Q \leq 1000$ \\ \hline
  $2$    & $10$       & $W_i=1$ för alla $i$. \\ \hline
  $3$    & $10$       & $|a-b|=1$ för alla par av $(a,b)$ där det finns en väg mellan $a$ och $b$, det vill säga är trädet en linje. \\ \hline
  $4$    & $25$       & För alla par av $(a,b)$ där det finns en väg mellan $a$ och $b$, gäller det att $\lfloor\frac{max(a,b)}{2} \rfloor=min(a,b)$.
  Med andra ord är trädet ett fullt binärt träd. \\ \hline
  $5$    & $15$       & $N \leq 50\,000$ \\ \hline
  $6$    & $30$       & Inga ytterligare begränsningar. \\ \hline
\end{tabular}

\section*{Exempelfall}
Exempelfall 3 uppfyller kraven för grupp 3 och exempelfall 4 uppfyller kraven för grupp 4.
