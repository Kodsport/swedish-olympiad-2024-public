\problemname{Trafikverket's Plan}

\noindent
During the winter, far too many accidents occur in Lönnköping due to icy roads.
Because the municipality of Lönnköping wanted to appear even more efficient than the cat
kingdom Katen's conquest (see \textit{Segment Monitoring}), their road network has very
few roads. The $N$ houses only have $N-1$ roads between them in total. Each road directly
connects two houses. If you are at a certain house, you can travel to another house using
road connected to the house you are at. The network is also designed such that one can
travel between every pair of houses by traversing one or more roads. The traffic department
has devised a brilliant plan to reduce the number of accidents: they intend to make each road
in the network one-way, so that everyone can commute to work in the morning. In the evening,
they will simply reverse all directions so that everyone can return home.

The only problem is that they are unsure whether or not their plan will work. What if some
residents cannot get to work in the morning? The traffic department has compiled a list of
where each resident lives and works and now wants you to determine, for each resident, whether
they can commute to work given their new plan. Their plan consists of the description of
the road network and the direction each road should have.

\section*{Input}
The first line of the input contains the integer $N$ ($1 \le N \le 3 \cdot 10^5$), the number of
houses in Lönnköping.

The following $N-1$ lines each contain the integers $a,b$ ($1 \leq a,b \leq N$, $a \neq b$),
meaning that there is a one-way road from house $a$ to house $b$. It is
guaranteed that the the undirected edges form a tree.

The next line contains the integer $Q$ ($1 \leq Q \leq 10^5$), the number of residents of Lönnköping.

Afterwards, $Q$ lines following, each containing the integers $H_i,W_i$ ($1 \leq H_i,W_i \leq N$),
the index of the house where resident $i$ lives and works respectively.

\section*{Output}
For each resident, print ``ja'' if they can get to work under the new plan. Otherwise,
print ``nej''. Print these in the same order of people as in the input.

\section*{Points}
Your solution will be tested on several test case groups.
To get the points for a group, it must pass all the test cases in the group.

\noindent
\begin{tabular}{| l | l | p{12cm} |}
  \hline
  \textbf{Group} & \textbf{Point value} & \textbf{Constraints} \\ \hline
  $1$    & $10$       & $N, Q \leq 1000$ \\ \hline
  $2$    & $10$       & $W_i=1$ for all $i$. \\ \hline
  $3$    & $10$       & $|a-b|=1$ for all pairs of $(a,b)$ such that there exists a road between $a$ and $b$, that is to say that the tree is a line. \\ \hline
  $4$    & $25$       & For all pairs $(a,b)$ such that there exists a road between $a$ and $b$, it holds that $\lfloor\frac{max(a,b)}{2} \rfloor=min(a,b)$.
  In other words, the tree is a full binary tree. \\ \hline
  $5$    & $15$       & $N \leq 50\,000$ \\ \hline
  $6$    & $30$       & No additional constraints. \\ \hline
\end{tabular}

\section*{Samples}
Sample 3 satisfies the constraints for subtask 3 and sample 4 satisfies the constraints for subtask 4.
