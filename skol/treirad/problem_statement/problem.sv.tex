\problemname{Tre i rad}
\noindent

Matematikern Lenore Oljer har gjort så mycket matte att hon har tröttnat på primtal. 
Hon har nu börjat studera så kallade ``trevliga tripletter''. 
Det är väldigt lätt att skapa en trevlig triplett. 
Börja med ett heltal $a$ som är
större än noll och skapa sedan tripletten $(a,a+1,a+2)$. 
Alla tripletter som skapats på detta sätt är trevliga. 
Några exempel på trevliga tripletter är $(4,5,6)$ och $(15,16,17)$.

För att lära sig mer om trevliga tripletter vill Lenore hitta hur 
många tal som kan skrivas som produkten av
alla talen i en trevlig triplett. Hon kallar dessa tal ``treiga tal''. 
Några exempel på treiga tal är $24=2 \cdot 3 \cdot 4$
och $336 = 6 \cdot 7 \cdot 8$. Lenore vill nu veta hur många treiga tal det 
finns som är mindre än talet $N$.

Skriv ett program som läser in talet $N$ och skriver ut hur många tal som är mindre än
$N$ och är treiga.

\section*{Indata}
Den första och enda raden av indata innehåller heltalet $N$ ($1 \le N \le 10^9$), som beskrivits ovan.


\section*{Utdata}
Skriv ut ett heltal, antalet treiga tal som är mindre än $N$.


\section*{Poängsättning}
Din lösning kommer att testas på en mängd testfallsgrupper.
För att få poäng för en grupp så måste du klara alla testfall i gruppen.

\noindent
\begin{tabular}{| l | l | p{12cm} |}
  \hline
  \textbf{Grupp} & \textbf{Poäng} & \textbf{Gränser} \\ \hline
  $1$    & $20$       & $N \leq 30$ \\ \hline
  $2$    & $20$       & $N \leq 80$ \\ \hline
  $3$    & $20$       & $N \leq 1000$ \\ \hline
  $4$    & $20$       & $N \leq 10^5$ \\ \hline
  $5$    & $20$       & Inga ytterligare begränsningar. \\ \hline
\end{tabular}
