\problemname{Reduplikation}
\noindent

Reduplikation är när man upprepar ett ord flera gånger, ofta för att förstärka eller ändra betydelsen.
I vissa språk som indonesiska är reduplikation väldigt vanligt, och där har det många olika
grammatiska funktioner. Även i svenska används reduplikation ibland, till exempel är det vanligt
att man säger \texttt{"hej hej"} istället för \texttt{"hej"}.

Rama är en flitig användare av reduplikation när han chattar med sina kompisar. Han skriver ibland samma ord
upp till nio gånger, vilket tar lång tid. Därför brukar han istället skriva ordet en gång, åtföljt av en
siffra som visar hur många gånger ordet ska upprepas.

Skriv ett program som läser in ett ord som Rama tänker upprepa
och hur många gånger det ska upprepas. 
Programmet ska sedan skriva ut ordet upprepat rätt antal gånger.

\section*{Indata}
Den första raden innehåller en sträng $s$ ($2 \leq |s| \leq 10$), ordet Rama vill skriva.
Ordet består av bokstäver från \texttt{a}-\texttt{z}.

Nästa rad innehåller en siffra mellan $1$ och $9$, antalet gånger som Rama vill skriva ordet.

\section*{Utdata}
Skriv ut $s$ så många gånger som siffran indikerar. 


\section*{Poängsättning}
Din lösning kommer att testas på en mängd testfallsgrupper.
För att få poäng för en grupp så måste du klara alla testfall i gruppen.

\noindent
\begin{tabular}{| l | l | p{12cm} |}
  \hline
  \textbf{Grupp} & \textbf{Poäng} & \textbf{Gränser} \\ \hline
  $1$    & $20$       & Siffran är $1$. \\ \hline
  $2$    & $80$       & Inga ytterligare begränsningar. \\ \hline
\end{tabular}
