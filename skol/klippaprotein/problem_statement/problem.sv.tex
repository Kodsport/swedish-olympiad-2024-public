\problemname{Klippa protein}
\noindent

Din kompis Björn har nyligen påbörjat ett stort kemiexperiment, med målet att skapa
ett hemligt protein. Alla protein beskrivs med strängar av bokstäver från
\texttt{a}-\texttt{z}, exempelvis \texttt{bcaa} och \texttt{hejsan}. Han behöver skapa
ett flertal så kallade \textit{rena} proteiner. Dessa är proteiner som består av kopior
av en och samma bokstav. För att göra detta har han mikroskopiska saxar som kallas CRISPR.
Med hjälp av dessa kan han ta bort upp till $K$ bokstäver som ligger bredvid varandra.
Med andra ord kan han välja ett intervall med upp till $K$ bokstäver att klippa bort.
Om ett protein delas i två delar när det klipps kommer dessa delar sedan att sättas ihop där de klipptes av. 

Eftersom han är väldigt lat vill han ha din hjälp att skriva ett program som beräknar
det minsta antalet gånger han behöver klippa ett visst protein för att det ska bli rent.
Notera att proteinet \textbf{inte} får bli en tom sträng, eftersom det isåfall inte
längre kan användas för kemi.


\section*{Indata}
Den första raden innehåller en sträng $P$ ($1 \leq |P| \leq 50$), proteinet Björn vill göra rent.
$P$ består endast av bokstäver från \texttt{a}-\texttt{z}.

På nästa rad följer ett heltal $K$ ($1 \leq K \leq |P|$), antalet bokstäver Björn kan ta bort med en klippning. 

\section*{Utdata}
Skriv ut ett heltal: det minsta antalet klippningar Björn behöver göra för att proteinet ska bli rent.


\section*{Poängsättning}
Din lösning kommer att testas på en mängd testfallsgrupper.
För att få poäng för en grupp så måste du klara alla testfall i gruppen.

\noindent
\begin{tabular}{| l | l | p{12cm} |}
  \hline
  \textbf{Grupp} & \textbf{Poäng} & \textbf{Gränser} \\ \hline
  $1$    & $20$       & $K = 1$ \\ \hline
  $2$    & $20$       & Proteinet består endast av bokstäverna \texttt{a} och \texttt{b}. \\ \hline
  $3$    & $60$       & Inga ytterligare begränsningar. \\ \hline
\end{tabular}

\section*{Förklaring av exempelfall}
Exempelfall $1$: ett möjligt sätt är att först klippa bort bokstäverna \texttt{pe}.
Vi får då strängen \texttt{exemlfall}. \\
Sedan kan vi klippa bort \texttt{fa} och får \texttt{exemlll}. \\
Till sist klippa vi bort \texttt{exem} och får \texttt{lll}, ett rent protein. 



\begin{figure}[h]
  \centering
  \includegraphics[width=0.7\textwidth]{sample2.PNG}
    \\Bilden visar lösningen till exempelfall 2. Om vi klipper bort alla \texttt{b}
    på sättet som bilden visar så blir proteinet rent på två klippningar.
  
\end{figure}
