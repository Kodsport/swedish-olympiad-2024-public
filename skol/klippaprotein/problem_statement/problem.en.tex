\problemname{Cutting Proteins}
\noindent

Your friend Björn has recently started a large chemistry experiment with the
goal of creating a secret protein. All proteins are described with strings
of letters from \texttt{a}-\texttt{z}, for example, \texttt{bcaa} and
\texttt{hello}. He needs to create several so-called \textit{pure} proteins.
These are proteins consisting of copies of the same letter. To do this, he has
microscopic scissors called CRISPR. With these, he can remove up to $K$
adjacent letters. In other words, he can choose an interval with up to $K$
letters to cut out. If a protein is divided into two parts when it is cut,
these parts will then be joined together where they were cut.

Since he is very lazy, he wants your help to write a program that calculates
the minimum number of times he needs to cut a specific protein to make it pure.
Note that the protein \textbf{must not} become an empty string, because in that
case it can no longer be used for chemistry.

\section*{Input}
The first line of input contains a string $P$ ($1 \leq |P| \leq 50$), the protein
that Björn wants to make pure. $P$ only consists of characters from \texttt{a}-\texttt{z}.

The next line contains the integer $K$ ($1 \leq K \leq |P|$), the number of characters
that Björn can remove in each cut.

\section*{Output}
Print an integer: the smallest number of cuts Björn needs to make in order to make the protein pure.

\section*{Points}
Your solution will be tested on several test case groups.
To get the points for a group, it must pass all the test cases in the group.

\noindent
\begin{tabular}{| l | l | p{12cm} |}
  \hline
  \textbf{Group} & \textbf{Point value} & \textbf{Constraints} \\ \hline
  $1$    & $20$       & $K=1$ \\ \hline
  $2$    & $20$       & The protein only consists of the characters \texttt{a} and \texttt{b}. \\ \hline
  $3$    & $60$       & No additional constraints. \\ \hline
\end{tabular}

\section*{Explanation of samples}

Sample $1$: one optimal way is to first cut the characters \texttt{pe}.
This results in the string \texttt{exemlfall}. \\
We can then cut away \texttt{fa}, resulting in \texttt{exemlll}. \\
Finally, we cut away \texttt{exem} and are left with \texttt{lll}, a pure protein. 



\begin{figure}[h]
  \centering
  \includegraphics[width=0.7\textwidth]{sample2.PNG}
    \\The picture shows the solution to sample 2. If we cut away all \texttt{b}
    in the way that the picture describes, the protein becomes pure in two cut.
  
\end{figure}
