\problemname{Trafikverkets misstag}
\noindent
Det otänkbara har hänt! Trafikverkets plan\footnote{Se https://open.kattis.com/problems/trafikverketsplan} gick åt skogen. Precis innan folk tänkte åka hem från jobbet i den lilla staden Lönnköping slog en hiskelig snöstorm ut
allt internet i staden. Eftersom internet är nere kan de inte få den senaste datan om Trafikverkets plan och kan därför inte
åka hem utan att riskera att krocka.
Det faller nu på dig att hitta ett sätt för alla att åka hem utan krockar. Lönnköpings vägnätverk består av $N$ hus och $N-1$ vägar. Varje väg går direkt mellan två hus.
Om du befinner dig vid ett visst hus kan du åka till alla hus som är kopplade till ditt nuvarande hus med en väg. Nätverket är dessutom konstruerat så att
man kan ta sig mellan varje par av hus genom att färdas längs med en eller flera vägar.

Tack vare datainsamlingen som skedde inför skapandet av Trafikverkets plan vet du var varje person befinner sig
(de är alla på jobbet och du vet var de jobbar) och vilket hus de vill åka hem till. Just nu sitter varje person i sin bil utanför sitt jobb och blockerar andra bilar från att köra förbi det huset.
För att få hem alla säkert kommer du att använda en drönare som flyger runt och säger till folk ``kör hem''. Då kommer de köra hem längs med den kortaste vägen mellan deras arbetsplats
och deras hem utan att stanna. När de väl är hemma kör de in i garaget och blockerar inga vägar längre. På grund av kylan är drönaren så långsam att du kan anta
att varje person hinner åka hem och parkera innan drönaren hinner fram till nästa person.

Om en bil någonsin börjar åka hem och det finns en bil i vägen kommer de att krocka på grund av det hala väglaget. Avgör om det finns en ordning som bilarna kan
köra hem som är helt fri från krockar. Om det finns en sådan ordning, hitta den.

\section*{Indata}
Den första raden i indatan innehåller två heltal, $N$ och $C$ ($1 \le C \leq N \le 2 \cdot 10^5$), antalet hus i Lönnköpings vägnätverk
och antalet personer som vill åka hem. 

De följande $N-1$ raderna innehåller vardera heltalen $a,b$ ($1 \leq a,b \leq N$, $a \neq b$),
vilket betyder att det finns en väg mellan hus $a$ och hus $b$. Det är garanterat att man kan färdas
mellan varje par av hus genom att åka längs med en eller flera vägar.

Därefter följer $C$ rader som vardera heltalen $W_i$ och $H_i$ ($1 \leq W_i,H_i \leq N$), vilket hus
den $i$:te personen jobbar i respektive vilket hus de bor i. Det är garanterat att som mest en person befinner sig vid varje hus ($W_i \neq W_j$ för varje $1 \leq i,j \leq N$, $i \neq j$).

\section*{Utdata}
Om det finns en ordning som invånarna kan åka hem utan att krocka, skriv först ut \texttt{``Yes''}. 
Skriv sedan ut ordningen som invånarna ska köra hem i. Invånaren som kommer först i indatan representeras
av $1$ i ordningen, den andra av $2$ och så vidare.

Om det inte finns en ordning utan krockar, skriv ut \texttt{``No''}.

\section*{Poängsättning}
Din lösning kommer att testas på en mängd testfallsgrupper.
För att få poäng för en grupp så måste du klara alla testfall i gruppen.

\noindent
\begin{tabular}{| l | l | p{12cm} |}
  \hline
  \textbf{Grupp} & \textbf{Poäng} & \textbf{Gränser} \\ \hline
  $1$    & $5$        & $C \leq 8, N \leq 100$ \\ \hline
  $2$    & $10$       & $N \leq 100$ \\ \hline
  $3$    & $10$       & $N \leq 4000$, längden på den längsta vägen är mindre än $50$. \\ \hline
  $4$    & $25$       & $N \leq 4000$ \\ \hline
  $5$    & $50$       & Inga ytterligare begränsningar. \\ \hline
\end{tabular}

%\section*{Exempelfall}

