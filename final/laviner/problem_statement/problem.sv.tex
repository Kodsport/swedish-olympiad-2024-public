\problemname{Laviner}
\noindent
Efter att Joshua såg igenom Alexanders försök att förfalska bilder på 
snöflingor\footnote{Se https://open.kattis.com/problems/falskflinga}
har Alexander blivit hämdlysten och besatt av snö.
Alexander har nu tänkt ut en ondskefull plan som kan innebära slutet för Joshua,
där det första steget är att bjuda med honom på en skidresa.

Tillsammans befinner de sig nu i ett stort skidsystem med $N$ skidbyar numrerade från $1$ till $N$.
Skidbyarna är sammankopplade med $N-1$ enkelriktade skidbackar.
Högst upp i skidsystemet ligger skidby $1$,
och från den går det att ta sig till alla andra
skidbyar genom att åka ner för en eller flera backar.
In till varje annan skidby leder exakt en backe.

Steg två i Alexanders ondskefulla plan är att 
starta en lavin i skidsystemet.
Det planerar han att göra genom att ställa sig i någon skidby och skapa oljud.
Då kommer snön i denna skidby att rasa ner längs alla backar som utgår från byn.
Sedan kommer snön att rasa vidare neråt i alla riktningar så långt som möjligt.
Lavinens \emph{skadogörelse} är antalet skidbyar som drabbas.

Joshua, som har märkt att Alexander inte verkar vara vid sina sinnens fulla bruk,
har genomskådat den ondskefulla planen.
Joshua är, som alla vet, händig, och kan välja vissa skidbyar att bygga murar i.
En mur i en viss skidby gör att lavinen inte kan sprida sig till byn.
Alexander kan inte heller starta lavinen från en murad skidby.

Men tiden är knapp, och efter att Joshua varnat
alla skidåkare för vad som är på väg att hända
hinner han bara bygga murar i $K$ olika skidbyar.
Joshua vet inte vilken skidby Alexander kommer ställa sig och skapa oljud i,
men han vill minimera antalet skidbyar som drabbas av lavinen.

Joshua kommer först att bygga de $K$ murarna så att den maximala skadegörelse
Alexander kan åstadkomma minimeras.
Sedan kommer Alexander skapa oljud i skidbyn där det orsakas maximal skadegörelse.
Din uppgift är att beräkna hur stor skadegörelsen blir,
alltså hur många skidbyar som drabbas av lavinen.

\section*{Indata}
Den första raden i indatan innehåller två heltal, $N$ och $K$ ($1 \le K < N \le 10^5$), 
antalet skidbyar i skidsystemet, och antalet murar Joshua kan sätta ut.

Därefter följer en rad med $N-1$ heltal $a_1, a_2, ..., a_{N-1}$, där $1 \le a_i \le i$.
För varje $i = 1, 2, ..., N-1$ finns det en skidbacke från skidbyn med nummer $a_i$ till skidbyn med nummer $i+1$. 

\section*{Utdata}
Skriv ut ett heltal: Den maximala skadogörelsen Alexander kan åstadkomma, givet att Joshua placerar murarna optimalt.


\section*{Poängsättning}
Din lösning kommer att testas på en mängd testfallsgrupper.
För att få poäng för en grupp så måste du klara alla testfall i gruppen.

\noindent
\begin{tabular}{| l | l | p{12cm} |}
  \hline
  \textbf{Grupp} & \textbf{Poäng} & \textbf{Gränser} \\ \hline
  $1$    & $10$       & $a_i = i$ för alla $(1 \le i \le N-1)$. \\ \hline
  $2$    & $10$       & $K = 1, N \le 1000$ \\ \hline
  $3$    & $10$       & $N \le 20$ \\ \hline
  $4$    & $15$       & $K = 1$ \\ \hline
  $5$    & $25$       & $N \leq 1000$ \\ \hline
  $6$    & $30$       & Inga ytterligare begränsningar. \\ \hline
\end{tabular}

%\section*{Exempelfall}

