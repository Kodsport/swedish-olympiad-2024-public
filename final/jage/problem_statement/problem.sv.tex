\problemname{Jage}
På Kemigården brukar $N$ chalmerister träffas varje dag och leka jage.
Leken går ut på att en person är \emph{jagare}, 
och när denne sedan \emph{tar} en annan person genom att nudda personen blir den tagna personen jagare istället.
Efter att ha varit med och lekt några gånger har du insett att något inte står rätt till.
Du har nämligen märkt att det plötsligt finns många fler jagare än vad det borde.
Om alla håller sig till reglerna borde det alltid finnas exakt en jagare,
men häromdagen gick leken överstyr och du hade plötsligt ett dussin blodtörstiga datastudenter efter dig.

Du har insett att det måste vara så att vissa studenter tar andra trots att de inte själva är jagare.
Nu vill du lista ut vilka studenter det är som fuskar på det här sättet.
En student fuskar ifall de tar någon när de vet att de själva inte är jagare.
Som tur är var du väldigt noggrann igår, och förde protokoll över både vilka som deltog i leken, och vem som tog vem.
Efter att ha samlat in denna information ska du nu skriva ett program som berättar vilka studenter som har fuskat.

\section*{Indata}
Den första raden innehåller två heltal $N$ och $M$ ($2 \leq N \leq 10^5$, $1 \leq M \leq 10^5$), 
antalet studenter som deltog i leken
och antalet gånger någon student tog en annan.

Därefter följer en rad med de mellanslagsseparerade namnen $s_1, ..., s_N$ på de studenter som deltog i leken.
Varje namn består av 1-20 tecken, och innehåller endast bokstäverna \texttt{a-z}. 
Alla namn är unika.
Det första namnet i denna lista är personen som började som jagare.

Därefter följer $M$ rader, där varje rad är på formen ``$s_i$ \texttt{tar} $s_j$'', där $s_i \neq s_j$.
Dessa anger i kronologisk ordning vilka studenter som tar varandra.
Eftersom du kollade väldigt noga vet du att du inte missade någon tagning.

\section*{Utdata}
Skriv först ut ett heltal, antalet studenter som fuskade.
Skriv därefter på en ny rad ut namnen på de studenter som fuskade, 
sorterade i alfabetisk ordning.

\section*{Poängsättning}
Din lösning kommer att testas på en mängd testfallsgrupper.
För att få poäng för en grupp så måste du klara alla testfall i gruppen.

\noindent
\begin{tabular}{| l | l | p{12cm} |}
  \hline
  \textbf{Grupp} & \textbf{Poäng} & \textbf{Gränser} \\ \hline
  $1$    & $10$       & $M = 1$. \\ \hline
  $2$    & $15$       & $N = 2$ \\ \hline
  $3$    & $15$       & \texttt{joshua} är med i leken, och om någon fuskar så är det han. \\ \hline
  $4$    & $20$       & $N \leq 200$. \\ \hline
  $5$    & $40$       & Inga ytterligare begränsningar. \\ \hline
\end{tabular}

\section*{Förklaring av testfall 1}
Från början är det \texttt{olle} som är jagare. 
Det betyder att \texttt{alexander} fuskar när han tar \texttt{joshua}. 
Efter det tror både \texttt{olle} och \texttt{joshua} att de är jagare.
Eftersom \texttt{joshua} tror att han är jagare fuskar han inte när han tar \texttt{alexander}.
Till slut tar \texttt{olle} \texttt{joshua}, och det är då \texttt{joshua} och \texttt{alexander} som tror att de är jagare.
Slutsatsen är att endast \texttt{alexander} fuskade.
